\documentclass[english]{article}
\usepackage[T1]{fontenc}
\usepackage[latin9]{inputenc}
\usepackage{babel}
\usepackage{cite}
\usepackage{url}

% For block comments
\usepackage{verbatim}

% For paper size and margin manipulation
\usepackage[a4paper, margin=1in]{geometry}

\begin{document}

\title{Project Proposal: Evaluating Politeness toward Computers}
\author{Matthew Neal, Shaown Sarker}
\maketitle

Clifford Nass and colleagues' 1999 study on human computer interaction looked to determine whether humans treat computers with politeness or not~\cite{nass1999}. Nass found that the social normative bias towards the human interviewer also extended to a computer program based interviewer, although objectively interviewees should not have had such a bias towards a non-human interviewer. The explanation of this behavior was that humans do not differentiate between computers and people when it comes to social interactions, and that the normative bias towards a human interviewer also extends to a computer one.

In this study, our primary motivation will be to focus on students from two disciplines within the North Carolina State University community. One group will come from Computer Science undergraduate and/or graduate students, and the other will likely be a group of non-Computer Science students. The idea with two distinct and separate groups of participants is to have a group of students that are intimately aware of the inner-workings of computer programs and a group of students that are much more likely to have access to computers from a layman rather than expert perspective. The underlying goal here is to determine whether the different level of understanding of computers translates differently to the personification that was found by Nass and colleagues back in 1999

In the study, Nass explained that humans simply reuse the same mental construct without much cognitive effort for social interactions with computers instead of forming a new one. However, in the intervening 16 years since the study, the sheer number of people with computer access, both at home and workplace, in the US has increased by a significant degree. Census reports indicate around 51\% of households had a computer at home in 1999~\cite{newburger2001}, while that percentage had increased to 83\% by 2013~\cite{file2014}. Furthermore, smart-phones have become completely ubiquitous over the same time period, the census bureau indicates that 63\% of households had a "handheld computer"(smart-phone, or other wireless computer) in 2013~\cite{file2014}. 

A secondary motivation for us will be what kind of effect this availability and extended familiarity of computers can have on the interviewer bias, which has not been studied in depth. It seems fair to conclude that a replication of the study, to determine whether the ingratiation towards the computer interviewer perceived by the participants in the original study is influenced by the degree of knowledge of computers, is a worthwhile endeavor.  The prevalence of available computers and smart-phones as a major aspect of our study will be impacted by whether or not we are able to obtain the recruitment procedures that Nass used in is original studies.  If that information is unavailable we will turn all of our attention to our primary motivation.

Methodologically we plan to recreate the study in as much detail as possible. Several nuances of this proposed study compared to the original are as follows:

\begin{itemize}
    \item We intend to use a web-based setup instead of the individual computer based program for the participants, which can be accessed by both their personal computers and/or smart-phones.This way we can have much more flexibility with how we administer the study. One possible threat to validity with this change is that if students use their own machines they may well have more of an attachment, for lack of a better word, for that machine which could affect the degree of personification. 
    
    \item In the interview, the participants are presented with a list of random facts, exactly same for all participants and later a quiz based on the facts presented. After the participants have completed the quiz, they provide feedback on the tutoring capability of the program via three different means - filling out a feedback form on the same computer, in a paper-pencil form, and on a different computer than the one used to take the quiz. In the original study, the tutor program expressed remarks praising itself to elicit positive feedback from the participants, which the authors considered a threat to the validity of their study. We will not include any such self-praise to manipulate the feedback of the participants.
    
    \item The original study contained a part where participants were subject to a voice based program alongside a textual one to emphasize human characteristics of the computer interviewer. We believe that a voice based component would be out of scope for our study given the time requirements that such a study would require.  Additionally, given that Nass' results were similar between the two phases of his study, the assumption is that we are not losing a significant part of the study by decreasing its scope in this way.
\end{itemize}

At the moment of writing this proposal, our hypothesis is that we will find varying level of personalization between the two groups of participants -- the group with CS background is likely to display more objectivity and thus be less ingratiating to the interviewer program in their feedback compared to the non-CS background group. If our hypothesis holds, then it will add to the finding of Nass' and colleagues that the knowledge of underlying workings of computers does indeed effect the social interaction between a human and a computer.

\bibliography{cited}{}
\bibliographystyle{plain}


\end{document}